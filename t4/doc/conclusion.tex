%-------------------------------------------------------------------------------------------------------
%-------------------------------------------------------------------------------------------------------
% Sec & Label

\section{Conclusion}
\label{sec:conclusion}

%-------------------------------------------------------------------------------------------------------
%-------------------------------------------------------------------------------------------------------
% Text


In order to perform theoretical and simulational analysis of the circuit Octave and Ngspice were used, respectively.


Theoretical methods were used to compute the gain, impedances and frequency responde of both of the stages. Contrary to past lab assignments the theoretical results differ a lot from the simulated results.

Comparing the frequency response graphs we can see that they are quite different. The most noticeable difference is that the theoretical method does not predict the higher cutoff frequency. This is explained by the fact that the theoretical method considers all the components to be ideal when in reality all of the components have some inductive characteristics which become noticeable at really high frequencies. In adition it is possible that the BJT model simulates the speed at which a transistor can be responsive.

The impedance and gain also differ differ significantly.

In conclusion, our simulated circuit was able to acheive a decent gain and merit value and so we considered it a sucess. In addition we were able to understand the functioning principles of a class A amplifier but also were exposed to the difficulty of choosing parameters that optimize the results of complex circuits.
