\section{Simulation Analysis}
\label{sec:simulation}

%-----------------------------------------------------------------------
%-----------------------------------------------------------------------
% Intro

In this section, Circuit T1 is reproduced with the help of Ngspice.

Firstly, the outcome of the simulation is shown, as well as a brief explanation
on how it was achived. Afterwards, a comparison is done between those values and
the ones attained in Subsection \ref{subsec:res_ana}.



%-----------------------------------------------------------------------
%-----------------------------------------------------------------------
% 			     Results - subsec
% ----------------------------------------------------------------------
% ----------------------------------------------------------------------

\subsection{Simulated results}
\label{sim_res}


% ----------------------------------------------------------------------
% Text

Table ~\ref{tab:op} shows the simulated operating point results for the circuit
under analysis. Compared to the theoretical analysis results, one notices the
following differences: describe and explain the differences.


% ----------------------------------------------------------------------
% Table - OP

\begin{table}[h]
	\centering
	\begin{tabular}{|l|r|}
		\hline    
		{\bf Name} & {\bf Value [A or V]} \\ \hline
    		i(vaux) & 9.187358e-04\\ \hline
i(h1) & 1.202281e-04\\ \hline
@g1[i] & -2.95727e-04\\ \hline
@id[current] & 1.038964e-03\\ \hline
@r1[i] & -2.82220e-04\\ \hline
@r2[i] & -2.95727e-04\\ \hline
@r3[i] & 1.350709e-05\\ \hline
@r4[i] & -1.20096e-03\\ \hline
@r5[i] & -1.33469e-03\\ \hline
@r6[i] & 9.187358e-04\\ \hline
@r7[i] & -9.18736e-04\\ \hline
n1 & 4.226624e+00\\ \hline
n2 & 4.830792e+00\\ \hline
n3 & 5.114025e+00\\ \hline
n4 & 4.871651e+00\\ \hline
n5 & 8.979579e+00\\ \hline
n6 & -1.84920e+00\\ \hline
n7 & -1.84920e+00\\ \hline
n8 & -2.78625e+00\\ \hline
v(n4,n2) & 4.085937e-02\\ \hline
v(n4,n8) & 7.657904e+00\\ \hline

	\end{tabular}
	
	\caption{Values given by Ngspice. Variables identified with a '$@$' have a
  	corresponding value in Ampere (A). The others are expressed in Volts (V).}
    
\label{tab:op}
\end{table}


%-----------------------------------------------------------------------
%-----------------------------------------------------------------------
% 			     Comp - subsec
% ----------------------------------------------------------------------
% ----------------------------------------------------------------------

\subsection{Values comparison}

For this comparison, note that

With all that considered, we observe that all the absolute values displayed in Table
\ref{tab:op} are identical to the ones shown in Table \ref{tab:oct}.

