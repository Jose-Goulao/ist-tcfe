\section{Simulation Analysis}
\label{sec:simulation}

%-----------------------------------------------------------------------
%-----------------------------------------------------------------------
% Intro

In this section, Circuit T1 is reproduced with the help of Ngspice.

Firstly, the outcome of the simulation is shown, as well as a brief explanation
on how it was achived. Afterwards, a comparison is done between those values and
the ones attained in Section \ref{sec:analysis}.



%-----------------------------------------------------------------------
%-----------------------------------------------------------------------
% 			     Results - subsec
% ----------------------------------------------------------------------
% ----------------------------------------------------------------------

\subsection{Simulated results}
\label{subsec:sim_res}


% ----------------------------------------------------------------------
% Text

Ngspice is a simulator for eletronic circuits that can output a variety of results.
This emulator computes the voltages in every node, as well as the potential difference
between two given nodes. Apart from that, the group made use of the command
{\em .options savecurrents} wich also enables the output of the currents that pass
trough all branches.

With the limitation that Ngspice only provides the current in the components and not through
the nodes, an aditional voltage source ($Vaux$) was added so that the current in $R_6$ ($I_c$)
is known. This source (not displayed in Figure \ref{fig:Desenho_t1}) as a voltage of 0V and it 
was implemented between $R_6$ and $R_7$. Therefore an aditional node had to be added (node $N7$).

As previously stated, $I_b$ is refered to as $G_1$. This is because, in Ngspice, a
voltage-controlled current source is identified with capital 'g' ($G$). In the case of
$V_c$, all current-controlled voltage source are identified with $H$.

Table \ref{tab:op} shows the simulated operating point results for Circuit T1.


% ----------------------------------------------------------------------
% Table - OP

\begin{table}[h]
	\centering
	\begin{tabular}{|l|r|}
		\hline    
		{\bf Name} & {\bf Value [A or V]} \\ \hline
    		@cb[i] & 0.000000e+00\\ \hline
@ce[i] & 0.000000e+00\\ \hline
@q1[ib] & 7.022567e-05\\ \hline
@q1[ic] & 1.404513e-02\\ \hline
@q1[ie] & -1.41154e-02\\ \hline
@q1[is] & 5.765392e-12\\ \hline
@rc[i] & 1.411536e-02\\ \hline
@re[i] & 1.411536e-02\\ \hline
@rf[i] & 7.022567e-05\\ \hline
@rs[i] & 0.000000e+00\\ \hline
v(1) & 0.000000e+00\\ \hline
v(2) & 0.000000e+00\\ \hline
base & 2.254108e+00\\ \hline
coll & 5.765392e+00\\ \hline
emit & 1.411536e+00\\ \hline
vcc & 1.000000e+01\\ \hline

	\end{tabular}
	
	\caption{Values provided by Ngspice. Variables identified with a '$@$' have a
  	corresponding value in Ampere (A). The others are expressed in Volts (V).}
    
\label{tab:op}
\end{table}



%-----------------------------------------------------------------------
%-----------------------------------------------------------------------
% 			     Comp - subsec
% ----------------------------------------------------------------------
% ----------------------------------------------------------------------

\subsection{Comparison}

With all that was previously considered, we observe that all the absolute values displayed
in Table \ref{tab:op} are identical to the ones shown in Table \ref{tab:oct}.

All the voltages in every node match with high precision. Moreover, $V_b$ and $V_c$ are
equal to the simulated values, wich are presented in Table \ref{tab:op} as $v(n2,n4)$ and
$v(n4,n8)$, respectively. Finally, theoretical $I_d$ is also the same as the one obtained
by Ngspice ('$@g1[i]$'). \\


It is also worth noting that all theoretical calculations consider every element of the
circuit to be ideal (without energy loss nor self-inductance nor any other phenomena that could
alter the results). Similarly, Ngspice also considers all components to be ideal. Therefore
every source of discrepancies between theoretical and simulated results are removed (apart from
the small limitations concerning calculations and the rounding of values).



