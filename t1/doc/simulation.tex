\section{Simulation Analysis}
\label{sec:simulation}

%-----------------------------------------------------------------------
%-----------------------------------------------------------------------
% Intro

In this section, Circuit T1 is reproduced with the help of Ngspice.

Firstly, the outcome of the simulation is shown, as well as a brief explanation
on how it was achived. Afterwards, a comparison is done between those values and
the ones attained in Subsection \ref{subsec:res_ana}.



%-----------------------------------------------------------------------
%-----------------------------------------------------------------------
% 			     Results - subsec
% ----------------------------------------------------------------------
% ----------------------------------------------------------------------

\subsection{Simulated results}
\label{sim_res}


% ----------------------------------------------------------------------
% Text

Table ~\ref{tab:op} shows the simulated operating point results for the circuit
under analysis. Compared to the theoretical analysis results, one notices the
following differences: describe and explain the differences.


% ----------------------------------------------------------------------
% Table - OP

\begin{table}[h]
	\centering
	\begin{tabular}{|l|r|}
		\hline    
		{\bf Name} & {\bf Value [A or V]} \\ \hline
    		@cb[i] & 0.000000e+00\\ \hline
@ce[i] & 0.000000e+00\\ \hline
@q1[ib] & 7.022567e-05\\ \hline
@q1[ic] & 1.404513e-02\\ \hline
@q1[ie] & -1.41154e-02\\ \hline
@q1[is] & 5.765392e-12\\ \hline
@rc[i] & 1.411536e-02\\ \hline
@re[i] & 1.411536e-02\\ \hline
@rf[i] & 7.022567e-05\\ \hline
@rs[i] & 0.000000e+00\\ \hline
v(1) & 0.000000e+00\\ \hline
v(2) & 0.000000e+00\\ \hline
base & 2.254108e+00\\ \hline
coll & 5.765392e+00\\ \hline
emit & 1.411536e+00\\ \hline
vcc & 1.000000e+01\\ \hline

	\end{tabular}
	
	\caption{Values given by Ngspice. Variables identified with a '$@$' have a
  	corresponding value in Ampere (A). The others are expressed in Volts (V).}
    
\label{tab:op}
\end{table}


%-----------------------------------------------------------------------
%-----------------------------------------------------------------------
% 			     Comp - subsec
% ----------------------------------------------------------------------
% ----------------------------------------------------------------------

\subsection{Values comparison}

For this comparison, note that

With all that considered, we observe that all the absolute values displayed in Table
\ref{tab:op} are identical to the ones shown in Table \ref{tab:oct}.

