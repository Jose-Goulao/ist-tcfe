%-------------------------------------------------------------------------------------------------------
%-------------------------------------------------------------------------------------------------------
% Sec & Label

\section{Theoretical Analysis}
\label{sec:analysis}


%-------------------------------------------------------------------------------------------------------
%-------------------------------------------------------------------------------------------------------
% Intro

In this section, the circuit in Figure \ref{fig:Desenho_t1} is analysed theoretically.

Two methods were used and both will be presented and explained. In Subsection \ref{subsec:mesh_met}
the aplication of the mesh method and its results are shown. In Subsection \ref{subsec:node_met} the
same is done with the  node method.



%-----------------------------------------------------------------------
%-----------------------------------------------------------------------
% 		     	    Mesh - subsec
% ----------------------------------------------------------------------
% ----------------------------------------------------------------------

\subsection{Mesh method}
\label{subsec:mesh_met}


% ----------------------------------------------------------------------
% Text

% ----------------------------------------------------------------------
% Eqs

% ----------------------------------------------------------------------
% Table - OCTAVE

\begin{table}[h]
	\centering
	\begin{tabular}{|l|r|}
    		\hline    
    		{\bf Name} & {\bf Value [A or V]} \\ \hline
    		\input{Table_Mesh-OCT}
  	\end{tabular}
  	\caption{Values computed by Octave. Variables identified with a '$@$' have a
  	corresponding value in Ampere (A). The others are expressed in Volts (V).}
 
\label{tab:oct}
\end{table}


%-----------------------------------------------------------------------
%-----------------------------------------------------------------------
% 			     Node - subsec
% ----------------------------------------------------------------------
% ----------------------------------------------------------------------

\subsection{Node method}
\label{subsec:node_met}

% ----------------------------------------------------------------------
% Text

% ----------------------------------------------------------------------
% Eqs

% ----------------------------------------------------------------------
% Table - OCTAVE

\begin{table}[h]
	\centering
	\begin{tabular}{|l|r|}
    		\hline    
    		{\bf Name} & {\bf Value [A or V]} \\ \hline
    		$V_{N1}$ & 4.226624e+00 \\ \hline 
$V_{N2}$ & 4.830792e+00 \\ \hline 
$V_{N3}$ & 5.114025e+00 \\ \hline 
$V_{N4}$ & 4.871651e+00 \\ \hline 
$V_{N5}$ & 8.979579e+00 \\ \hline 
$V_{N6}$ & -1.849204e+00 \\ \hline 
$V_{N8}$ & -2.786253e+00 \\ \hline 
$V_{b}$ & 4.085937e-02 \\ \hline 
$V_{c}$ & 7.657904e+00 \\ \hline 
$@I_{b}$ & 2.957272e-04 \\ \hline 
$@I_{c}$ & 9.187358e-04 \\ \hline 
$@I_{H1}$ & -1.202281e-04 \\ \hline
  	\end{tabular}
  	\caption{Values computed by Octave. Variables identified with a '$@$' have a
  	corresponding value in Ampere (A). The others are expressed in Volts (V).}
 
\label{tab:oct}
\end{table}


