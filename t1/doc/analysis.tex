%-------------------------------------------------------------------------------------------------------
%-------------------------------------------------------------------------------------------------------
% Sec & Label

\section{Theoretical Analysis}
\label{sec:analysis}


%-------------------------------------------------------------------------------------------------------
%-------------------------------------------------------------------------------------------------------
% Intro

In this section, the circuit in Figure \ref{fig:Desenho_t1} is analysed theoretically.

Two methods were used and both will be presented and explained. In Subsection \ref{subsec:mesh_met}
the aplication of the mesh method and its results are shown. In Subsection \ref{subsec:node_met} the
same is done with the  node method. \\

In both of these methods, three important equations were used: both Kirchhoff's laws (Kirchhof's
current law - eq(\ref{eq:kcl}) and Kirchhoff's voltage law - eq(\ref{eq:kvl})); Ohm's law (eq(\ref{eq:ohm})).

The algebraic sum of all the currents in any given node is zero:
\begin{equation}
	\sum I_i = 0
	\label{eq:kcl}
\end{equation}

The algebraic sum of all the voltages in any given closed-loop circuit (mesh) is zero:
\begin{equation}
	\sum V_i = 0
	\label{eq:kvl}
\end{equation}

The potential difference between the two nodes connected to a resistor is equal to the current that 
passes through the resistor multiplied by the its resistance.
\begin{equation}
	V_i = R_iI_i
	\label{eq:ohm}
\end{equation}



%-----------------------------------------------------------------------
%-----------------------------------------------------------------------
% 		     	    Mesh - subsec
% ----------------------------------------------------------------------
% ----------------------------------------------------------------------

\subsection{Mesh method}
\label{subsec:mesh_met}




\begin{table}[h]
	\centering
	\begin{tabular}{|l|r|}
    		\hline    
    		{\bf Name} & {\bf Value [A or V]} \\ \hline
    		$@I_{1}$ & 2.822201e-04 \\ \hline 
$@I_{2}$ & 2.957272e-04 \\ \hline 
$@I_{3}$ & 9.187358e-04 \\ \hline 
$@I_{4}$ & -1.038964e-03 \\ \hline 
$V_{b}$ & -4.085937e-02 \\ \hline 
$V_{c}$ & -7.657904e+00 \\ \hline 
$@I_{b}$ & -2.957272e-04 \\ \hline 
$@I_{c}$ & -9.187358e-04 \\ \hline 

  	\end{tabular}
  	\caption{Values computed by Octave. Variables identified with a '$@$' have a
  	corresponding value in Ampere (A). The others are expressed in Volts (V).}
 
\label{tab:mesh}
\end{table}



%-----------------------------------------------------------------------
%-----------------------------------------------------------------------
% 			     Node - subsec
% ----------------------------------------------------------------------
% ----------------------------------------------------------------------

\subsection{Node method}
\label{subsec:node_met}


Similarly to the previous subsection, for the node method, ground was considered to be where it is identified in
Figure \ref{fig:Desenho_t1}. In adition, assume $V_{Ni}$ to be the voltage in node $Ni$ (every node position can
also be found in Figure \ref{fig:Desenho_t1}).

The results were computed by Octave and organized in Table \ref{tab:node}

\begin{table}[h]
	\centering
	\begin{tabular}{|l|r|}
    		\hline    
    		{\bf Name} & {\bf Value [A or V]} \\ \hline
    		$V_{N1}$ & 4.226624e+00 \\ \hline 
$V_{N2}$ & 4.830792e+00 \\ \hline 
$V_{N3}$ & 5.114025e+00 \\ \hline 
$V_{N4}$ & 4.871651e+00 \\ \hline 
$V_{N5}$ & 8.979579e+00 \\ \hline 
$V_{N6}$ & -1.849204e+00 \\ \hline 
$V_{N8}$ & -2.786253e+00 \\ \hline 
$V_{b}$ & 4.085937e-02 \\ \hline 
$V_{c}$ & 7.657904e+00 \\ \hline 
$@I_{b}$ & 2.957272e-04 \\ \hline 
$@I_{c}$ & 9.187358e-04 \\ \hline 
$@I_{H1}$ & -1.202281e-04 \\ \hline
  	\end{tabular}
  	\caption{Values computed by Octave. Variables identified with a '$@$' have a
  	corresponding value in Ampere (A). The others are expressed in Volts (V).}
 
\label{tab:node}
\end{table}


