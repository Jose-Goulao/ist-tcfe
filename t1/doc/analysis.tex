\section{Theoretical Analysis}
\label{sec:analysis}

%-----------------------------------------------------------------------
%-----------------------------------------------------------------------
% Intro


In this section, the Circuit T1 is analysed theoretically.

A precise description of the procedure used to compute all the values is presented.
Furthermore, the equations that were aplied and the attained results are also shown.


%-----------------------------------------------------------------------
%-----------------------------------------------------------------------
% 		     Description and Eqs - subsec
% ----------------------------------------------------------------------
% ----------------------------------------------------------------------

\subsection{Methodology}

% ----------------------------------------------------------------------
% Text

% ----------------------------------------------------------------------
% Eqs

\begin{equation}
	Ri(t) + v_O(t) = v_I(t).
	\label{eq:kvl}
\end{equation}


%-----------------------------------------------------------------------
%-----------------------------------------------------------------------
% 			     Results - subsec
% ----------------------------------------------------------------------
% ----------------------------------------------------------------------

\subsection{Obtained results} 





% ----------------------------------------------------------------------
% Table - OCTAVE

\begin{table}[h]
	\centering
	\begin{tabular}{|l|r|}
    		\hline    
    		{\bf Name} & {\bf Value [A or V]} \\ \hline
    		$V_{b}$ & -4.752955e+00 \\ \hline 
$V_{c}$ & 7.657904e+00 \\ \hline 
$@I_{b}$ & -2.957272e-01 \\ \hline 
$@I_{c}$ & 9.187358e-01 \\ \hline 
$@I_{d}$ & 1.038964e+00 \\ \hline
  	\end{tabular}
  	
  	\caption{Values computed by Octave}
 
\label{tab:teste}
\end{table}


