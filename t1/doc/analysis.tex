%-------------------------------------------------------------------------------------------------------
%-------------------------------------------------------------------------------------------------------
% Sec & Label

\section{Theoretical Analysis}
\label{sec:analysis}


%-------------------------------------------------------------------------------------------------------
%-------------------------------------------------------------------------------------------------------
% Intro

In this section, the circuit in Figure \ref{fig:Desenho_t1} is analysed theoretically.

Two methods were used and both will be presented and explained. In Subsection \ref{subsec:mesh_met}
the aplication of the mesh method and its results are shown. In Subsection \ref{subsec:node_met} the
same is done with the  node method.



%-----------------------------------------------------------------------
%-----------------------------------------------------------------------
% 		     	    Mesh - subsec
% ----------------------------------------------------------------------
% ----------------------------------------------------------------------

\subsection{Mesh method}
\label{subsec:mesh_met}


% ----------------------------------------------------------------------
% Text

% ----------------------------------------------------------------------
% Eqs



%-----------------------------------------------------------------------
%-----------------------------------------------------------------------
% 			     Node - subsec
% ----------------------------------------------------------------------
% ----------------------------------------------------------------------

\subsection{Node method}
\label{subsec:node_met}

% ----------------------------------------------------------------------
% Table - OCTAVE

\begin{table}[h]
	\centering
	\begin{tabular}{|l|r|}
    		\hline    
    		{\bf Name} & {\bf Value [A or V]} \\ \hline
    		$V_{b}$ & -4.752955e+00 \\ \hline 
$V_{c}$ & 7.657904e+00 \\ \hline 
$@I_{b}$ & -2.957272e-01 \\ \hline 
$@I_{c}$ & 9.187358e-01 \\ \hline 
$@I_{d}$ & 1.038964e+00 \\ \hline
  	\end{tabular}
  	\caption{Values computed by Octave. Variables identified with a '$@$' have a
  	corresponding value in Ampere (A). The others are expressed in Volts (V).}
 
\label{tab:oct}
\end{table}


