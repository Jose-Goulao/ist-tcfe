\section{Theoretical Analysis}
\label{sec:analysis}

%----------------------------------------------------------------------
%----------------------------------------------------------------------
% Intro


In this section, the Circuit T1 is analysed theoretically.

A precise description of the procedure used to compute all the values is presented.
Furthermore, the equations that were aplied and the attained results are also shown.


%----------------------------------------------------------------------
%----------------------------------------------------------------------
% Description and Eqs

\subsection{Methodology}

The circuit consists of a single V-R-C loop where a current $i(t)$ circulates. The
voltage source $v_I(t)$ drives its input, and the output voltage $v_O(t)$ is taken from
the capacitor terminals. Applying the Kirchhoff Voltage Law (KVL), a single
equation for the single loop in the circuit can be written as

\begin{equation}
  Ri(t) + v_O(t) = v_I(t).
  \label{eq:kvl}
\end{equation}

%----------------------------------------------------------------------
%----------------------------------------------------------------------
% Results

\subsection{Obtained results} 

\begin{table}[h]
  \centering
  \begin{tabular}{|l|r|}
    \hline    
    {\bf Name} & {\bf Value [A or V]} \\ \hline
    $V_{b}$ & -4.752955e+00 \\ \hline 
$V_{c}$ & 7.657904e+00 \\ \hline 
$@I_{b}$ & -2.957272e-01 \\ \hline 
$@I_{c}$ & 9.187358e-01 \\ \hline 
$@I_{d}$ & 1.038964e+00 \\ \hline
  \end{tabular}
  \caption{Values computed by Octave}
  \label{tab:teste}
\end{table}


