%-------------------------------------------------------------------------------------------------------
%-------------------------------------------------------------------------------------------------------
% Sec & Label

\section{Theoretical Analysis}
\label{sec:analysis}


%-------------------------------------------------------------------------------------------------------
%-------------------------------------------------------------------------------------------------------
% Intro

In this section, the circuit in Figure \ref{fig:Desenho_t1} is analysed theoretically.

Two methods were used and both will be explained and presented. In Subsection \ref{subsec:mesh_met}
the aplication of the mesh method and its results are shown. In Subsection \ref{subsec:node_met} the
same is done with the node method. \\


For both methods, ground was considered to be where it is identified in Figure \ref{fig:Desenho_t1}.
In adition, assume $V_{Ni}$ to be the voltage in node $Ni$ (every node position can also be found in
Figure \ref{fig:Desenho_t1}). Furthermore, assume $V_{N7}$ is equal to $V_{N6}$, the full explanation
can be found in Subsection \ref{subsec:sim_res}. \\

In both of these methods, three important equations were used: both Kirchhoff's laws (Kirchhof's
current law (KCL) - eq.(\ref{eq:kcl}) and Kirchhoff's voltage law (KVL) - eq.(\ref{eq:kvl}));
Ohm's law (eq.(\ref{eq:ohm})).

The algebraic sum of all the currents in any given node is zero:
\begin{equation}
	\sum I_i = 0
	\label{eq:kcl}
\end{equation}

The algebraic sum of all the voltages in any given closed-loop circuit (mesh) is zero:
\begin{equation}
	\sum V_i = 0
	\label{eq:kvl}
\end{equation}

The potential difference between the two nodes connected to a resistor is equal to the current that 
passes through the resistor multiplied by its resistance.
\begin{equation}
	V_i = R_iI_i
	\label{eq:ohm}
\end{equation}



%-----------------------------------------------------------------------
%-----------------------------------------------------------------------
% 		     	    Mesh - subsec
% ----------------------------------------------------------------------
% ----------------------------------------------------------------------

\subsection{Mesh method}
\label{subsec:mesh_met}

To correclty use the mesh method, firstly, four currents must be considered, one for each simple mesh.
They were identified as follows: $I_1$ - associated with the top left mesh; $I_2$ - associated with
the top right mesh; $I_3$ - associated with the bottom left mesh; $I_4$ - associated with the bottom
right mesh. Each one of these currents is assumed to run counterclockwise. \\

Afterwards, eq.(\ref{eq:kvl}) needs to be applied in the meshes not containing any type of current
sources (eq.(\ref{eq:kvl_m1}) and eq.(\ref{eq:kvl_m3})) . Moreover, it is essential to relate the
remaining mesh current with those created by the current sources (eq.(\ref{eq:rel_cur1}) and
eq.(\ref{eq:rel_cur2})). Likewise, eq.(\ref{eq:rel_cur3}) is also obtained.

\begin{equation}
	V_a = I_1(R_1+R_3+R_4)-I_2(R_3)-I_3(R_4)
	\label{eq:kvl_m1}
\end{equation}

\begin{equation}
	-V_c = I_3(R_4+R_6+R_7)-I_1(R_4)
	\label{eq:kvl_m3}
\end{equation}

\begin{equation}
	I_2 = -I_b
	\label{eq:rel_cur1}
\end{equation}

\begin{equation}
	I_4 = -I_d
	\label{eq:rel_cur2}
\end{equation}

\begin{equation}
	I_3 = -I_c
	\label{eq:rel_cur3}
\end{equation}

Further relations need to be composed in order to solve the circuit. We must not forget the two equations
from the linear dependent sources (eq.(\ref{eq:depsrc1}) and eq.(\ref{eq:depsrc2})). In adition, by
making use of eq.(\ref{eq:ohm}), one more equation is acquired, eq.(\ref{eq:rel_ohm1}).

\begin{equation}
	I_b = K_bV_b
	\label{eq:depsrc1}
\end{equation}

\begin{equation}
	V_c = K_cI_b
	\label{eq:depsrc2}
\end{equation}

\begin{equation}
	I_1(R_3)-I_2(R_3) = V_b
	\label{eq:rel_ohm1}
\end{equation}

With eight equations and eight unknown variables, the system can be solved.
The results were computed by Octave and organized in Table \ref{tab:mesh}

\begin{table}[h]
	\centering
	\begin{tabular}{|l|r|}
    		\hline    
    		{\bf Name} & {\bf Value [A or V]} \\ \hline
    		$@I_{1}$ & 2.822201e-04 \\ \hline 
$@I_{2}$ & 2.957272e-04 \\ \hline 
$@I_{3}$ & 9.187358e-04 \\ \hline 
$@I_{4}$ & -1.038964e-03 \\ \hline 
$V_{b}$ & -4.085937e-02 \\ \hline 
$V_{c}$ & -7.657904e+00 \\ \hline 
$@I_{b}$ & -2.957272e-04 \\ \hline 
$@I_{c}$ & -9.187358e-04 \\ \hline 

  	\end{tabular}
  	\caption{Values computed by Octave. Variables identified with a '$@$' have a
  	corresponding value in Ampere (A). The others are expressed in Volts (V).}
 
\label{tab:mesh}
\end{table}



%-----------------------------------------------------------------------
%-----------------------------------------------------------------------
% 			     Node - subsec
% ----------------------------------------------------------------------
% ----------------------------------------------------------------------

\subsection{Node method}
\label{subsec:node_met}

The node method uses KCL (eq.(\ref{eq:kcl})) in conjunction with Ohm’s law (eq.(\ref{eq:ohm})) to define
equations that when solved give the voltage value of each node in relation to ground (Node 0, $V_0 = 0$).
In this circuit we defined six equations that equate the currents entering a particular node with the
currents leaving said node. 

In order to have equations that solve for the node’s voltage, a relation between current and voltage is made using 
Ohm’s law (given a resistance between two nodes, the current that passes the resistance can be written as 
$I=\frac{V_2-V_1}{R_1}$)

To simplify the equations it is useful to use the conductance $G_n$ which is the inverse of the resistance $R_n$ 
($G_n=\frac{1}{R_n}$)

\begin{equation}
	G_2(V_1-V_2)+G_1(V_3-V_2) - G3(V_2-V_4) = 0
	\label{}
\end{equation}

\begin{equation}
	G_2(V_1-V_2)+K_b(V_4-V_2) = 0
	\label{}
\end{equation}

\begin{equation}
	G3(V_2-V_4)+G_5(V_5-V_4)-G_4V_4)+I_c = 0
	\label{}
\end{equation} 

\begin{equation}
	K_b(V_2-V_4)-G_5(V_5-V_4) = -I_d
	\label{}
\end{equation}

\begin{equation}
	G_7(V_6-V_8)-I_c = I_d
	\label{}
\end{equation}

\begin{equation}
	G_6(-V_6)-G_7(V_6-V_8) = 0
	\label{}
\end{equation}

In order to solve the circuit two more equations are used to relate the voltage difference between
the nodes that are conected to the voltage sources.

\begin{equation}
	V_3 = V_a
	\label{}
\end{equation}

\begin{equation}
	K_c\frac{(-V_6)}{G_6}-(V_4-V_8) = 0
	\label{}
\end{equation}


With these eight equations it is possible to solve the system using Octave.
The results were organized in Table \ref{tab:node}

\begin{table}[h]
	\centering
	\begin{tabular}{|l|r|}
    		\hline    
    		{\bf Name} & {\bf Value [A or V]} \\ \hline
    		$V_{N1}$ & 4.226624e+00 \\ \hline 
$V_{N2}$ & 4.830792e+00 \\ \hline 
$V_{N3}$ & 5.114025e+00 \\ \hline 
$V_{N4}$ & 4.871651e+00 \\ \hline 
$V_{N5}$ & 8.979579e+00 \\ \hline 
$V_{N6}$ & -1.849204e+00 \\ \hline 
$V_{N8}$ & -2.786253e+00 \\ \hline 
$V_{b}$ & 4.085937e-02 \\ \hline 
$V_{c}$ & 7.657904e+00 \\ \hline 
$@I_{b}$ & 2.957272e-04 \\ \hline 
$@I_{c}$ & 9.187358e-04 \\ \hline 
$@I_{H1}$ & -1.202281e-04 \\ \hline
  	\end{tabular}
  	\caption{Values computed by Octave. Variables identified with a '$@$' have a
  	corresponding value in Ampere (A). The others are expressed in Volts (V).}
 
\label{tab:node}
\end{table}


